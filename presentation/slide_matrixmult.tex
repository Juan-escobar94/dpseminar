\section{Matrix-chain Multiplication Problem}


\begin{frame}{Matrix-chain Multiplication Problem}
  It is known that multiplying matrices is a very frequent mathematical
  operation among the most varied application fields. Multiplying matrices 
  $A_1$ and $A_2$ requires matching dimensions: the number of rows in $A_1$
  must match the number of columns in $A_2$ if one is to multiply the matrices 
  in the order $A_{1}A_{2}$.
  \\
  \vspace{2em}
  The product obtained by multiplying a chain of matrices $A_{1}A_{2}...A_{n}$ 
  can be obtained by parenthesizing the chain in many different ways.
\end{frame}


\begin{frame}{Matrix-chain Multiplication Problem - Parenthesization}
  \begin{block}{Fully Parenthesized Product}
    A product of matrices is called \textbf{\emph{fully parenthesized}}
    either if it is a single matrix, or a fully parenthesized product of two matrixes, 
    surrounded by parenthesis.
  \end{block}
  \pause
  For example, a matrix chain consisting of the matrices $A_{1}A_{2}A_{3}A_{4}$ can be
  parenthesized in the following ways:
  \begin{itemize}
    \item $(((A_{1}A_{2})A_{3})A_{4})$
    \item $(A_{1}((A_{2}A_{3})A_{4}))$
    \item $(A_{1}(A_{2}(A_{3}A_{4})))$
    \item $((A_{1}(A_{2}A_{3}))A_{4})$
    \item $((A_{1}A_{2})(A_{3}A_{4}))$
  \end{itemize}
\end{frame}

\begin{frame}[allowframebreaks]{Matrix Multiplication Problem - Problem Statement}
  The way we parenthesize a matrix can have an enormous impact in the number
  of multiplications we have to compute in the end. 
  
  Take for example matrices $A_1$, $A_2$ and $A_3$ with dimensions $3 \times 50$, $50 \times 2$
  and $2 \times 40$ respectively, forming the chain $A_{1}A_{2}A_{3}$. we have the following parenthesization choices
  \\
  \vspace{0.5em}
  \begin{itemize}
    \item $((A_{1}A_{2})A_{3})$ yielding $3 \times 50 \times 2 + 3 \times 2 \times 40 = 540$ multiplications.
    \item  $(A_{1}(A_{2}A_{3}))$ yielding $50 \times 2 \times 40 + 3 \times 50 \times 40 = 10000$ multiplications.
  \end{itemize}

  \framebreak

  \begin{block}{Abridged Problem Statement}
    Given a chain of matrices $A_{1}A_{2}...A_{n}$, fully parenthesize the chain in a way
    that minimizes the amount of multiplications to be computed.
  \end{block}
\end{frame}

\begin{frame}[allowframebreaks]{Matrix-chain Multiplication Problem - Naive approach}
  Trying to solve this problem with a complete search approach leads us to the question:
  \\
  \vspace{1em}
  In how many different ways can we parenthesize a chain of n matrices?

  \framebreak
  when $n = 1$, the chain consists of only one matrix which we can only parenthesize in one way.
  when $n \geq 2$ a fully parenthesized matrix chain is the product of to fully parenthesized
  subproducts, and the split between two products may occur between the $kth$
  and the $(k + 1)st$ matrices for any $k = 1, 2, ..., n - 1 $. obtaining:
    \[
      P(n) = \left\{\begin{array}{lr}
        1, & \text{if } n = 1,\\
        \sum_{k=1}^{n-1}P(k)P(n-k), & \text{if } n \geq 2.
        \end{array}\right\}
    \]
  
  The time complexity for this recurrence is exponential! ($O(2^n)$)
\end{frame}

\begin{frame}{Matrix Multiplication Problem - Optimal Solution Structure}
  \begin{block}{Definition}
   Denote the matrix $A_{i...j}$ as the matrix obtained from the product
   of multiplying matrices $A_{i}A_{i+1}...A_{j}$ \\
  \vspace{0.5em}
  For the non trivial problem, we must split the product between matrices $A_k$ and $A_{k+1}$,
  with $ i \leq k < j$. The cost of parenthesizing this way is the cost
  of computing the matrix $A_{i...k}$, the cost of computing $A_{k+1...j}$,
  and the cost of multiplying both matrices together
  \end{block}

  \pause
  \begin{alertblock}{Optimal Substructures}
    % drüber nachdenken wie man dies anders ausdrücken kann (kürzer)
    An optimal parenthesization is composed of two parenthesized matrix products as well,
    which in turn, also have to be optimal: if this weren't the case, then there must be another
    way to parenthesize this chain, meaning that the "global" parenthesization we claimed as optimal would not
    be optimal, a contradiction!
  \end{alertblock}
\end{frame}

\begin{frame}{Matrix Multiplication Problem - Recursive Solution}
  for a chain of matrices $A_{i...j}$ let us define a structure $m[i, j]$ as an array
  that stores the solution to all the subproblems that can arise
  along the process.
  \vspace{0.5em}
  the case $i = j$ is trivial, no multiplications are necessary %to compute the product of a single matrix.
  \\
  In the non trivial case, $i < j$, we make use of the structure of an optimal solution.
  We split the product at  $A_k$ and $A_{k+1}$
  \vspace{0.5em}
  %Defining the cost of splitting the product at matrices $A_k$ and $A_{k+1}$ as the optimal
  %cost of each product individually, plus the cost of multiplying both matrices together.
  \\
  We do not know the value of k that guarantees us the optimal solution, so we iterate over
  all possible $i - j$ values, yielding:   

  \[
  m[i,j] = \left\{\begin{array}{lr}
    0, & \text{if } i = j,\\
    min_{i \leq k < j} \{m[i,k] + m[k + 1, j] + p_{i-1}p_{k}p_{j}\} & \text{otherwise}
    \end{array}\right\}
  \]
\end{frame}

\begin{frame}{Matrix Multiplication Problem - Computing Optimal Costs}
  At this point we have the following choices:
  \begin{itemize}
    \item use a top-down approach, implement the recursive solution and add memoization.
    \item make use of a tabular, bottom-up approach with no necessity of recursion.
  \end{itemize}
  \vspace{0.7em}
  consider P to be a list describing the dimensions in the matrix chain, following the schema
  [$rows_{A_{1}}  \times columns_{A_{1}}(rows_{A_{2}}) \times columns_{A_{2}} (rows_{A_{3}}) ... columns_{A_{n}}$]:
  \\
  \vspace{0.7em}
  \begin{tabular}{r|ccc}
    matrix & $A_1$ & $A_2$ & $A_3$ \\
    \hline 
    dimensions & $\textcolor{blue}{3} \times \textcolor{green}{5}$ & $\textcolor{green}{5} \times \textcolor{pink}{2}$ & $\textcolor{pink}{2} \times \textcolor{orange}{3}$ \\
    P & [\textcolor{blue}{3}, \textcolor{green}{5}, \textcolor{pink}{2}, \textcolor{orange}{3}]
  \end{tabular}

\end{frame}

\begin{frame}[fragile]{Bottom-Up Approach}
  
  \begin{lstlisting}
    INF = 1 << 32
    
    def matrix_chain_order(p):
      n = len(p)
      m = [[INF] * (n-1) for x in range(n-1)] # store optimal costs value 
      s = [[0] * (n-1) for x in range(n-1)] # store k value to reconstruct solution
      for i in range(0, n-1):
        m[i][i] = 0
        for l in range(2, n):
          for i in range(0, n-l):
            j = i + l - 1
            for k in range(i, j):
              q = m[i][k] + m[k+1][j] + p[i]*p[k+1]*p[j+1]
              if (q < m[i][j]):
                m[i][j] = q
                s[i][j] = k
      return (m, s)
  \end{lstlisting}
\end{frame}

\begin{frame}{Matrix-chain Multiplication Problem: Example}
  \begin{exampleblock}{Example}
    consider the matrix chain described by P = [4, 5, 2, 3, 5]
  \end{exampleblock}

  \vspace{0.5em}

  $
    \begin{bmatrix}
      \infty & \infty & \infty & \infty \\
      \infty & \infty & \infty & \infty \\
      \infty & \infty & \infty & \infty \\
      \infty & \infty & \infty & \infty \\
    \end{bmatrix}
  $
  \pause
  $
    \begin{bmatrix}
      0 & \infty & \infty & \infty \\
      \infty & 0 & \infty & \infty \\
      \infty & \infty & 0 & \infty \\
      \infty & \infty & \infty & 0 \\
    \end{bmatrix}
  $
  \pause
   $
    \begin{bmatrix}
      0 & 40 & \infty & \infty \\
      \infty & 0 & 30  & \infty \\
      \infty & \infty & 0 & 30 \\
      \infty & \infty & \infty & 0 \\
    \end{bmatrix}
  $
  \pause
  $
    \begin{bmatrix}
      0 & 40 & 64 & \infty \\
      \infty & 0 & 30  & 80 \\
      \infty & \infty & 0 & 30 \\
      \infty & \infty & \infty & 0 \\
    \end{bmatrix}
  $
  \pause
   $
    \begin{bmatrix}
      0 & 40 & 64 & 110 \\
      \infty & 0 & 30  & 80 \\
      \infty & \infty & 0 & 30 \\
      \infty & \infty & \infty & 0 \\
    \end{bmatrix}
  $
\end{frame}


\begin{frame}[fragile]{Matrix Multiplication Problem - Printing the Optimal Solution}
  \begin{lstlisting}
    def printOptimal(s, i, j):
      if i == j:
          print("A_" + str(i+1), end="")
      else:
          print("(", end="")
          printOptimal(s, i, s[i][j])
          printOptimal(s, s[i][j]+1, j)
          print(")", end="")

    #print the result
    printOptimal(s, 0, 3)
  \end{lstlisting}

  Calling the code above to print the actual solution yields: \\
  \vspace{0.1em}
  \pause 
  $ s = \begin{bmatrix}
      0 & 1 & 2 & 2 \\
      0 & 0 & 2  & 2 \\
      0 & 0 & 0 & 3 \\
      0 & 0 & 0 & 0 \\
    \end{bmatrix}
    sol = ((A_{1}A_{2})(A_{3}A_{4}))$
\end{frame}