\section{Matrix-chain Multiplication Problem}


\begin{frame}{Matrix-chain Multiplication Problem}
  It is known that multiplying matrices is a very frequent mathematical
  operation among the most varied application fields. Multiplying Matrices 
  $A_1$ and $A_2$ requires matching dimensions: the number of rows in $A_1$
  must match the number of columns in $A_2$ if one is to multiply the Matrices 
  in this order $A_{1}A_{2}$.
  \\
  \vspace{2em}
  The product obtained by multiplying a chain of matrices $A_{1}A_{2}...A_{n}$ 
  can be obtained by parenthesizing the chain in many different ways.
\end{frame}


\begin{frame}{Matrix-chain Multiplication Problem - Parenthesization}
  \begin{block}{Fully Parenthesized Product}
    A product of matrices is called \textbf{\emph{fully parenthesized}}
    either if it is a single matrix, or a fully parenthesized product of two matrixes, 
    surrounded by parenthesis.
  \end{block}
  \pause
  For example, a matrix chain consisting of the matrices $A_{1}A_{2}A_{3}A_{4}$ can be
  parenthesized in the following ways:
  \begin{itemize}
    \item $(((A_{1}A_{2})A_{3})A_{4})$
    \item $(A_{1}((A_{2}A_{3})A_{4}))$
    \item $(A_{1}(A_{2}(A_{3}A_{4})))$
    \item $((A_{1}(A_{2}A_{3}))A_{4})$
    \item $((A_{1}A_{2})(A_{3}A_{4}))$
  \end{itemize}
\end{frame}

\begin{frame}[allowframebreaks]{Matrix Multiplication Problem - Problem Statement}
  The way we parenthesize a matrix can have an enormous impact in the number
  of multiplications we have to compute in the end. 
  
  Take for example matrices $A_1$, $A_2$ and $A_3$ with dimensions $3 \times 50$, $50 \times 2$
  and $2 \times 40$ respectively, forming the chain $A_{1}A_{2}A_{3}$. we have the following parenthesization choices
  \\
  \vspace{0.5em}
  \begin{itemize}
    \item $((A_{1}A_{2})A_{3})$ yielding $3 \times 50 \times 2 + 3 \times 2 \times 40 = 540$ multiplications.
    \item  $(A_{1}(A_{2}A_{3}))$ yielding $50 \times 2 \times 40 + 3 \times 50 \times 40 = 10000$ multiplications.
  \end{itemize}

  \framebreak

  \begin{block}{Abridged Problem Statement}
    Given a chain of matrices $A_{1}A_{2}...A_{n}$, fully parenthesize the chain in a way
    that minimizes the amount of multiplications to be computed.
  \end{block}
\end{frame}

\begin{frame}[allowframebreaks]{Matrix-chain Multiplication Problem - Naive approach}
  Trying to solve this problem with a complete search approach leads us to the question:
  \\
  \vspace{1em}
  In how many different ways can we parenthesize a chain of n matrices?

  \framebreak
  when $n = 1$, the chain consists of only one matrix which we can only parenthesize in one way.
  when $n \geq 2$ a fully parenthesized matrix chain is the product of to fully parenthesized
  subproducts, and the split between two products may occur between the $kth$
  and the $(k + 1)st$ matrices for any $k = 1, 2, ..., n - 1 $. obtaining:
    \[
      P(n) = \left\{\begin{array}{lr}
        1, & \text{if } n = 1,\\
        \sum_{k=1}^{n-1}P(k)P(n-k), & \text{if } n \geq 2.
        \end{array}\right\}
    \]
  
  The time complexity for this recurrence is exponential! ($O(2^n)$)
\end{frame}

\begin{frame}{Matrix Multiplication Problem - Optimal Solution Structure}
  \begin{block}{Definition}
   Denote the matrix $A_{i...j}$ as the matrix obtained from the product
   of multiplying matrices $A_{i}A_{i+1}...A_{j}$ \\
  \vspace{0.5em}
  For the non trivial problem, we must split the product between matrices $A_k$ and $A_{k+1}$,
  with $ i \leq k < j$. The cost of parenthesizing this way is the cost
  of computing the matrix $A_{i...k}$, the cost of computing $A_{k+1...j}$,
  and the cost of multiplying both matrices together
  \end{block}

  \pause
  \begin{alertblock}{Optimality}
    An optimal parenthesization is composed of two optimal parenthesized matrix products as well,
    if any of these two composing it are not optimal parenthesizations, our optimal parenthesization can not be optimal,
    a contradiction!
  \end{alertblock}
\end{frame}

\begin{frame}{Matrix Multiplication Problem - Recursive Solution}
  
    \[
    m[i,j] = \left\{\begin{array}{lr}
      0, & \text{if } i = j,\\
      min_{i \leq k < j} \{m[i,k] + m[k + 1, j] + p_{i-1}p_{k}p_{j}\} & \text{otherwise}
      \end{array}\right\}
    \]
  
\end{frame}

\begin{frame}{Matrix Multiplication Problem - Computing Optimal Costs}
  
\end{frame}

\begin{frame}{Matrix Multiplication Problem - Constructing an Optimal Solution}
  
\end{frame}