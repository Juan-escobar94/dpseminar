\section{Introduction}
\subsection{Definition}

\begin{frame}
  \frametitle{Introduction}
  \begin{block}{Definition}
   What is Dynamic Programming?
   \begin{itemize}
     \item Problem Solving Paradigm
     \item Algorithm Design Technique
   \end{itemize}
  \end{block}
\end{frame}

\begin{frame}
  \frametitle{Introduction}
  \begin{block}{Key Properties}
   A problem which is solvable through Dynamic
   Programming has the following properties:
   \begin{itemize}
     \item Optimal Substructure
     \item Overlapping Subproblems
   \end{itemize}
  \end{block}

  % \begin{alertblock}{}
  %   here is some thext
  % \end{alertblock}
\end{frame}

\subsection{Fibonacci Sequence}
\begin{frame}
  \frametitle{Introduction - Fibonacci Sequence}
  Let us first take a look at a simple Dynamic Programming approach.

  \begin{block}{Example}
    The Fibonacci sequence is defined as follows:
    \\
    \[
    fib(n) = \left\{\begin{array}{lr}
      1, & \text{for } n = 0, n = 1\\
      fib(n-1) + fib(n-2), & \text{otherwise}
      \end{array}\right\}
    \]
    \\
  \end{block}
\end{frame}



\begin{frame}[fragile]
  \frametitle{Introduction - Fibonacci Sequence}
  \begin{block}{Example}
    A naive solution for computing the n-th number in the 
    Fibonacci sequence:
  \end{block}
  \begin{lstlisting}

      def fibonacci(n):
        if (n <= 1):
          return 1
        else:
          return fibonacci(n - 1) + fibonacci(n - 2)

    \end{lstlisting}
\end{frame}


\begin{frame}
\frametitle{Introduction - Fibonacci Sequence}
\begin{figure}[ht]
  \centering
  \begin{tikzpicture}
    \node {6}
      child { node {5}
        child { node {4} 
          child{ node {3} 
            child { node {2}
              child {node {1}}
              child {node {0}}
            }
            child { node {1}}
          }
          child{ node[color=red] {2} }
        }
        child { node[color=red] {3} }
      }
      child { node[color=red] {4}
      };
  \end{tikzpicture}
  \caption{Fibonacci recursion tree with $n = 6$}
\end{figure}
\end{frame}

\begin{frame}
% \frametitle{Introduction - Fibonacci Sequence}
\begin{figure}[ht]
  \centering
  \begin{tikzpicture}
    \node {6}
      child { node {5}
        child { node {4} 
          child{ node {3} 
            child { node {2}
              child {node {1}}
              child {node {0}}
            }
            child { node {1}}
          }
          child{ node[color=red] {2} }
        }
        child { node[color=red] {3} }
      }
      child { node[color=red] {4}
      };
  \end{tikzpicture}
  \caption{Fibonacci recursion tree with $n = 6$}
\end{figure}
\begin{alertblock}{!}
  Every red-colored Node is a computation we are doing not for the
  first time!
\end{alertblock}
\end{frame}


\begin{frame}[fragile]
  \frametitle{Introduction - Fibonacci Sequence}
  \begin{block}{Dynamic Programming Approach}
    How about we keep record of the values we compute?
  \end{block}
  \begin{lstlisting}
    # initialize the look up table with 
    # non-recursively defined fibonacci numbers.
    look_up_table = {
      0: 1,
      1: 1  
    }

    def dp_fibonacci(n):
      if (look_up_table.at(n)):
        return look_up_table.at(n)
      else :
        look_up_table[n] = dp_fibonacci(n - 1) + dp_fibonacci(n - 2)
        return look_up_table[n]
  \end{lstlisting}
\end{frame}


\begin{frame}
  \begin{block}{Discussion}
    Our modifications to the original approach have
    caused changes in the following aspects in our program:
    \begin{itemize}
      \item Memory Usage
      \item Time needed to compute the solution
    \end{itemize}
  \end{block}
\end{frame}