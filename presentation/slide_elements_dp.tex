\section{Elements Of Dynamic Programming}

% \begin{frame}{Elements Of Dynamic Programming}
%   \begin{block}{Key Aspects}
%     \begin{enumerate}
%       \item Optimal Substructures
%     \end{enumerate}
%   \end{block}
% \end{frame}


\begin{frame}{Elements Of Dynamic Programming}
  \begin{block}{Key Aspects}
    \begin{enumerate}
      \item Optimal Substructures
    \end{enumerate}
    \pause
    The solution for a subproblem is part of the solution of the original problem
  \end{block}
\end{frame}

% \begin{frame}{Elements Of Dynamic Programming}
%   \begin{block}{Key Aspects}
%     \begin{enumerate}
%       \item Optimal Substructres
%       \item Overlapping Subproblems
%     \end{enumerate}
%   \end{block}
% \end{frame}


\begin{frame}{Elements Of Dynamic Programming}
  \begin{block}{Key Aspects}
    \begin{enumerate}
      \item Optimal Substructures
      \item Overlapping Subproblems
    \end{enumerate}
    \pause
    When a recursive algorithm repeatedly revisits the same subproblem in different branches along its recursion
    tree, we say that the problem has overlapping subproblems.
  \end{block}
\end{frame}


\begin{frame}{Elements of DP: Rod-cutting Problem Example}
  Let us examine this properties based on the following problem:
  \begin{block}{Rod-cutting Problem Abridged Problem Statement}
  Given a rod of length n, and a table of prices $p_i$ for $i = 1, 2, ..., n$,
determine the maximum revenue $r_n$ obtained by cutting the rod and selling the pieces
  \end{block}
  %format differently?
\end{frame}

\begin{frame}{Elements of DP: Rod-cutting Problem Example}
  With the following Price Table:
  \\
  \vspace{2em}
\begin{table}[ht]
  \centering
  \begin{tabular}{l|ll}
    \textbf{length i}&\textbf{price $p_i$}\\\hline
      1 & 2\\
      2 & 6\\
      3 & 7\\
      4 & 8\\
      5 & 9
  \end{tabular}
\end{table}
\end{frame}