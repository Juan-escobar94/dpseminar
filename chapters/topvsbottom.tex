\section{Top-down vs Bottom-up}

We have seen that two common ways to apply Dynamic Programming are using the top-down approach with memoization
and the bottom-up method. 

Top-down approaches are a natural transformation from the complete search recursion, meaning we take the recursive function
and adjust the computation to always perform a look up in the data structure first. Due to the recursive nature of the
top-down approach, problems that are continously revisited may lead to a big function call overhead, slowing down the 
process. 

On the other hand, bottom-up approaches have no overhead from recursive function calls, as they are not defined
in a recursive fashion. Instead, every single sub-problem from the bottom (e.g. the first Fibonacci numbers) is visitied, and 
in contrast to the top-down approach, they are visited only once. This style of problem solving is generally for most programmers
not as intuitive as the top-down approach.

Depending on the problem, either approach can have better performance. In some cases the top down approach may not cover the entire search space,
and visit, as just said, only the necessary subproblems to solve the main problem. It could also be that a sub problem is visited a tremendous amount of times
making the bottom up variant better, as the nested recursive call can end up beind costly in terms of overhead and computing time, likewise.s

