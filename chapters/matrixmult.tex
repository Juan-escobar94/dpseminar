\section{Matrix-Chain Multiplication}

The Matrix-Chain Multiplication problem is another example in which we can 
apply Dynamic Programming, yielding important performance gains.

Suppose we have a product (also called chain) of Matrices, numbered from 1 to n. Their product is
calculated as follows:

$$A_1A_2A_3...A_n$$

The number of scalar multiplications that have to be performed when multiplying matrices $A_1$
and $A_2$ with dimensions $m_1xn_1$ and $m_2n_2$ is equal to $m_1n_1n_2$ (remember that
for a two matrices, rows and columns must match in their number, else the matrices are not 
compatible and the multiplication cannot be performed). 

Now, how does this problem even relate to Dynamic Programming? remember that Matrix 
multiplication is \emph{associative} which means that parenthesization does not alter
the end result, in other words: $((A_1A_2)A_3) = (A_1((A_2A_3))$.

The product obtained by this multiplication will not vary with the way we parenthesize,
but the ammount of multiplications required may very well vary. Suppose we have
matrices of dimensions 3x50, 50x2 and 2x40, the first parenthesization yields 
3x50x2 + 3x2x40 = 540 multiplications, while the second variant yields
50x2x40 + 3x50x40 = 10000 multiplication, an almost scandalous difference.

\begin{quote}How we parenthesize a chain of matrices can have a dramatic impact of evaluating
the product \cite{cormen2009introduction}\end{quote}

the abridged problem statement can be written down as follows: given a n chain of matrices,
find a parenthesization with the least ammount of multiplications. These matrices are compatible
for multiplication, and for $i = 1, 2, ..., n$ the matrix $A_i$ has dimension $p_{i-1} * p_i$

Brute forcing this problem:


catalan numbers formula


Complexity


complexity new algo

new algo description

new algo graphics