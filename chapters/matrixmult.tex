\section{Matrix-Chain Multiplication}

The Matrix-Chain Multiplication problem is another example in which we can 
apply Dynamic Programming, yielding important performance gains.

Suppose we have a product (also called chain) of Matrices, numbered from 1 to n. Their product is
calculated as follows:

$$A_1A_2A_3...A_n$$

The number of scalar multiplications that have to be performed when multiplying matrices $A_1$
and $A_2$ with dimensions $m_1 \times n_1$ and $m_2\times n_2$ is equal to $m_1*n_1*n_2$ .Rremember that
for a two matrices, rows and columns must match in their number, else the matrices are not 
compatible and the multiplication can not be performed, hence we could have also written $m_1*m_2*n_2$

The product obtained by this multiplication will not vary with the way we parenthesize,
but the ammount of multiplications required may very well vary. Suppose we have
matrices of dimensions $A_1 = 3 \times 50$, $A_2 = 50 \times 2$ and $A_3 = 2 \times 40$, 
the first parenthesization $((A_1A_2)A_3)$ yields $3*50*2 + 3*2*40 = 540$
multiplications, while the second variant $(A_1((A_2A_3))$ yields 
$50*2*40 + 3*50*40 = 10000$ multiplications, an almost scandalous difference.
How we parenthesize a chain of matrices can have a dramatic impact of evaluating
the product \cite{cormen2009introduction}

Now, how does this problem relate to Dynamic Programming? as we just said, in matrix 
multiplication, the way we parenthesize does not alter the end product, in other words it is
\emph{associative}, meaning:  $((A_1A_2)A_3) = (A_1((A_2A_3))$. By using Dynamic Programming
we can abuse the nature of this problem (exhibits both optimal substructures and
overlapping subproblems, as we will see) and find an optimal parenthesization.

This problem clearly posseses the property of optimal substructures, an optimal 
parenthesized product consists as well of optimal parenthesized products. An optimal
parenthesized product would be the parenthesization resulting in the minimum of 540
multiplications in our previous example. Suppose instead of having matrices $A_1$, 
$A_2$ and $A_3$ we had products of matrices, which in turn must be optimally parenthesized
to yield an optimal solution.

the abridged problem statement can be written down as follows: given a n chain of matrices,
find a parenthesization with the least ammount of multiplications. These matrices are compatible
for multiplication, and for $i = 1, 2, ..., n$ the matrix $A_i$ has dimension $p_{i-1} * p_i$


In order to better understand the search space for this problem, let us first count the number
of ways we can parenthesize a matrix chain. in \cite{cormen2009introduction}, the number of
parenthesizations of a sequence of n matrices are denoted by P(n). When n = 1, there is just one matrix and there is only
one way to parenthesize it. when $n \geq 2$, a fully parenthesized matrix chain is the product of to 
fully parenthesized subproducts, and the split between two products may occur between the \emph{k}th and the
\emph{(k + 1)}st matrices for any k = 1, 2, ..., n-1. obtaining:
  \\
  \[
    P(n) = \left\{\begin{array}{lr}
      1, & \text{if } n = 1,\\
      \sum_{k=1}^{n-1}P(k)P(n-k), & \text{if } n \geq 2.
      \end{array}\right\}
  \]
  \\


The solution to this recurrence is $\Omega (2^n)$ \cite{cormen2009introduction}, for an input of n matrices.
going over all possible solutions using a brute force approach is a quite unefficient strategy.
