
 
 \section{introduction}


What is dynamic programming? it really sounds like one of those big buzzwords that seem to
attract big audiences that tend to follow trends. But put in simple words, Dynamic Programming is just a tabular method, 
which reduces a lot of duplicate computations on a problem.How does dynamic programming work? lets have a look at an example first: The fibonacci numbers,
which form a sequence defined as follows:

  \[
    fib(n) = \left\{\begin{array}{lr}
      n, & \text{for } n = 1, n = 2\\
      fib(n-1) + fib(n-2), & \text{otherwise}
      \end{array}\right\}
  \]

we could write a simple program to compute the nth fibonacci number as follows:

\begin{minted}[mathescape, linenos]{python}

def fibonacci(n):
  if (n <= 1):
    return n
  else :
    return fibonacci(n - 1) + fibonacci(n + 2)

\end{minted}


to get a rough picture of the space complexity for this program, we could depict the steps
the program would take to compute an arbitrary number n, in the form of a tree. For n = 6, we have:


\begin{figure}[ht]
  \centering
  \begin{tikzpicture}
    \node {6}
      child { node {5}
        child { node {4} 
          child{ node {3} 
            child { node {2}
              child {node {1}}
              child {node {0}}
            }
            child { node {1}}
          }
          child{ node[color=red] {2} }
        }
        child { node[color=red] {3} }
      }
      child { node[color=red] {4}
      };
  \end{tikzpicture}
  \caption{Fibonacci recursion tree with $n = 6$}
  \label{fig:fib1}
\end{figure}

There are 6 levels in this call tree and although the red nodes in \autoref{fig:fib1} are 
not expanded, the algorithm has still to do all the work and visit this nodes until it gets to a node
labeled with one or with zero. We can easily see how this can get out of hands with higher numbers, the 
number of nodes is $O(2^n)$
