\documentclass{article}


\begin{document}
\title{Dynamic Programming}
\author{Juan Andrés Osorio Escobar}
\date{\today}
\maketitle

\begin{abstract}
Dynamic Programming (short \textbf{\emph{DP}}) refers to a problem solving paradigm,
  which seeks to solve optimization and counting problems in a very efficient manner by 
  trading off memory space against increased time efficiency. This is made 
  possible by the traits that a DP problem must exhibit: 
  \begin{itemize}
    \item \textbf{Optimal substructures}: An optimal solution to the main problem contains within itself optimal solutions to subproblems \cite{cormen2009introduction}.
    \item \textbf{Ovelapping subproblems}: When a recursive algorithm repeatedly revisist the same subproblem in different branches along its recursion tree, we say that the optimization problem has overlapping subproblems \cite{cormen2009introduction}. This is the key characteristic
    which allows the search space for a problem to be drastically reduced \cite{halim2013competitive}.
  \end{itemize}
  
  In problems where DP applies, the run time to a problem can be reduced from exponential 
  time complexity (using complete search algorithms), to polynomial time complexity. DP algorithms are implemented
  in one of two ways: either recursively/top-down with memoization, or bottom-up. Both methods make use of a table in which the solution to subproblems are stored.
  We shall revisit also both techniques and analyze their pros and cons.

  % Dynamic programming (short \emph{DP}) refers to a problem solving paradigm,
  % in which the solution to a (most of the time) optimization (maximize
  % this, minimize that variable) problem is sought. One can imagine
  % that DP works similarly to divide and conquer: it divides the
  % problem into different subproblems, which it most of the time
  % recursively computes (when used in a top-down manner) and uses the partial
  % answers to reach an optimal solution. One of the key differences between DP and divide and conquer is
  % that instead of computing every single subproblem, DP makes use of the overlapping subproblems characteristic,
  % it allows dp to manage a global table
  % (usually called memo table) in which already computed solutions are
  % kept. Before doing any computing, the DP algorithm performs a
  % look-up on this table, and returns a value if there is one. if there
  % is no such value, DP proceeds to compute recursively the solution
  % and writes values on the table as it goes along, making them available
  % for further computations that may stumble upon the exact same subproblem.
  % the main idea behind this memo table is to
  % >>remember<< and never compute something twice. In many
  % mathematical/computational problems in which DP can be applied, DP
  % algorithms are often times much more efficient in regards to time complexity compared
  % to complete search algorithms, usually reducing exponential time complexity to polynomial, which
  % is an incredible gain.

  % The key aspect about DP is that the problem at hand must exhibit optimal
  % substructures, meaning problems must overlap, meaning that the solution to a subproblem
  % is dependent of the solution to its subsubproblems \cite{cormen2009introduction}.
  

  
\bibliographystyle{acm}
\bibliography{references.bib}
  
\end{abstract}


\end{document}
